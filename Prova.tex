\documentclass[12pt, letterpaper, twoside]{article}  % preamble, tipo del documento
\usepackage[utf8]{inputenc}  % encoding
\usepackage{parskip}  % per andare a capo, dice sia utile

\title{Prova}
\author{me stesso \thanks{fatto da me stesso ovviamente}}
\date{oggi, quindi marzo 2021}

\begin{document}  % sempre inizio e fine, This is known as the body of the document

\maketitle  % stampa il preamble nel body

\tableofcontents  % dice vada qui, così poi lo caccia all'inizio. Insomma l'indice


\section{Una sezione}
We have now added a title, author and date to our first \LaTeX{} document!
\\Tanto per separare dalla table of contents, si poi si scazzano tutti dopo perché questo l'ho aggiunto dopo, eh oh ci posso fa poco adesso visto è tardi e voglio far veloce. =)

% this is a commento, è in verde come python ;)
\textbf{this is bold}, \textit{this is italic}, and \underline{this is underline}, c'è anche \emph{this one}. Poi:
\begin{itemize}
  \item Questo è il primo punto;
  \item e questo il secondo
\end{itemize}

\begin{enumerate}
  \item Questo è un primo punto
  \item e questo è proprio un secondo punto
\end{enumerate}

\newpage\begin{abstract}
Questo è un abstract, dice che è per i testi scientifici, serve per l'introduzione, non mi sembra la cosa più utile del mondo
\end{abstract}

Dopo Questo si scrive il paragrafo.

Con due volte a capo si ha il secondo paragrafo, fa abbastanza cacare sto documento.
\\Comunque scrivo tanto giusto per vedere come viene sto documento eh, mica c'ho tanto da dire ;). Per vedere la pagina nuova più che altro...
\\Adesso organizziamo il documento in sezioni:  % si va a capo bene con \\

% \chapter{First Chapter} solo per book e report del tipo di documenti
\newpage\section{Introduction}  % questo l'ho scoperto da me :)))))
Prima sezione

\section{Second Section}
altra sezione

\subsection{First Subsection}
la 2.1

\subsection{Second Subsection}
un'altra ancora

\subsubsection{First Subsubsection}
sempre più sub

\paragraph{That's a Paragraph}
ci scrivo un po' di cose tanto per far vedere come funziona...
\\è un po'messo male qui ma non importa
\\si mi so rotto di mettere le maiuscole :)

\section*{Unnumbered Section}
questa non ha numero, per separare sai...

\section{Tabelle}
Qui ci devi mettere le tabelle quando te ne ricordi

\section{Un po'di math}
Scriviamo qualcosa neh:
$ciao = ciao$ anche \(ciao = ciao\)
\[ciao = ciao\]
se lo vuoi mettere a capo o meno $a_b$ e $a^b$
\[ \int_a^b x^2dx \]


\end{document}